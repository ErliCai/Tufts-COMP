
\documentclass[12pt]{article}

\usepackage{graphicx}
\graphicspath{ {./images/} }

\usepackage{epsfig}
\usepackage{amsmath,amsthm}
\usepackage{listings}


\newtheorem{lemma}{Lemma}
\newtheorem{theorem}{Theorem}


\usepackage{titlesec}
\titleformat{\section}
{\normalfont\Large\bfseries}{Question~\thesection:}{1em}{}

\newlength{\toppush}
\setlength{\toppush}{2\headheight}
\addtolength{\toppush}{\headsep}


\def\subjnum{Comp 170}
\def\subjname{Computation Theory}


\def\doheading#1#2#3{\vfill\eject\vspace*{-\toppush}%
  \vbox{\hbox to\textwidth{{\bf} \subjnum: \subjname \hfil Erli Cai}%
    \hbox to\textwidth{{\bf} Tufts University, Fall 2020 \hfil#3\strut}%
    \hrule}}


\newcommand{\htitle}[1]{\vspace*{1.25ex plus 1ex minus 0ex}%
\begin{center}
{\large\bf #1}
\end{center}} 



\begin{document}
\doheading{2}{title}{Homework 02}

\setlength{\parindent}{0pt}

\section{Close}
a. $A = \{a\}^* $ for some $ a\in \Sigma$\\

b. $A = \Sigma^*$\\

c. We know A is regular, so we can assume that we have a DFA $M = \{Q,\Sigma,\delta,s,F \}$ such that L(M) = A. Now we want to construct NFA $M' = \{Q',\Sigma,\delta',s',F' \}$ where $L(M') = Close(A)$\\

Let 
\begin{flalign*}
Q' &= Q \times \{0,1\}\\
\delta'((p,0),a) &=  \begin{cases}
   (\delta(p, a),0) &  \textrm{ for all $p \in Q, a\in\Sigma$}\\
    (q, 1)   &\textrm{ for some  $q = \delta(p,b)$ where $  b\in\Sigma$ and $  q \ne   \delta((p, a) $}\\
   \end{cases}\\
\delta'((p,1),a) &=    (\delta(p, a),1)  \quad\textrm{ for all $p \in Q, a\in\Sigma$}\\
s' &= (s,0)\\
F' &= F\times\{1\}
\end{flalign*}

Suppose we have a string x in NFA $M'$. States $(p,0) \in Q \times \{0\}$ represent the state it ended up every transition follows transition function of NFA M. State $(q,1) \in Q \times \{1\}$ represent the state it ended up if exactly one transition does not follow transition function of $\delta$ NFA M.\\
It should be noted that in transition from $(p,0) \in Q \times \{0\}$ to $(q,1)\in Q \times \{1\}$, q should be reachable from p(This is because the result have to be close). Suppose we are taking character a in that transition, we would also have $q\ne\delta(p,a)$, this is because Close(A) does not necessarily include A.

\pagebreak

\section{Close But Not Quite}
Given a language A over alphabet $\Sigma$, \\
define $CloseButNot(A) = \{$x$|$x and are close, $y\in{A}, x\notin{A} \}$\\

Let Not(A) = $\{x|x\notin{A}\}$\\
Then, $CloseButNot(A) =  Not(A) \cap Close(A)$
Since we know that if A is regular then Close (A) is regular and intersection of two regular language is also regular. We only need to show that Not(A) is regular.\\

If A is regular, then we have DFA M =  $(Q,\Sigma, \delta,s,F)$ that accepts A.
Then  DFA $M'$ =  $(Q,\Sigma, \delta,s,F')$ where $F' = Q \backslash F$ accepts Not(A). This means that if A is regular language, then Not(A) is also regular language.\\

In conclusion, if A is regular language, then $CloseButNot(A)$ is also regular language.


\pagebreak
\section{All-NFA}

An all-NFA M is a 5-tuple $(Q,\Sigma, \delta,s,F)$ that accepts $x\in\Sigma^* $ if every possible state that M could be in after reading input x is a state from F.\\

We first prove that every all-NFA recognizes a regular language:\\

We construct DFA $M' = (Q',\Sigma, \delta',s',F')$ such that
\begin{flalign*}
Q' &= P(Q) \mbox{(power set of Q)} \\
\delta'(q',a) &=  \{ \delta(q,a) | q\in q' \} \mbox{ for all } a\in \Sigma, q'\in Q' \\
s' &= s\\
F' &= P(F)
\end{flalign*}

Claim: $L(M) = L(M')$\\
\begin{proof}
By definition we have $\delta'(q',a) =  \{ \delta(q,a) | q\in q' \} $ for all$ a\in \Sigma, q'\in Q' $\\
therefore,
\begin{equation}
 \hat\delta'(q',x) =  \{ \hat\delta(q,x) | q\in q' \} \mbox{ for all } x \in \Sigma^*, q'\in Q'
\end{equation}
\end{proof}
therefore, the following statements are equivalent:
\begin{flalign*}
x\in L(M) & \Leftrightarrow \{\hat\delta(s,x)\} \subseteq F \Leftrightarrow  \hat\delta'(s,x) \in F'  \Leftrightarrow  x\in L(M') \Leftrightarrow \mbox{x is a regular language}
\end{flalign*}


Now we prove every regular language is recognized by some all-NFA:\\

A DFA is a all-NFA, and every regular language is recognized by some DFA. Therefore every regular language is recognized by some all-NFA.\\

In conclusion, we have shown that  all-NFAs recognize the class of regular languages





\end{document}


