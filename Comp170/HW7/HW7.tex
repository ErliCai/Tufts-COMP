
\documentclass[12pt]{article}

\usepackage{graphicx}
\graphicspath{ {./images/} }

\usepackage{epsfig}
\usepackage{amsmath,amsthm}
\usepackage{listings}


\newtheorem{lemma}{Lemma}
\newtheorem{theorem}{Theorem}


\usepackage{titlesec}
\titleformat{\section}
{\normalfont\Large\bfseries}{Question~\thesection:}{1em}{}

\newlength{\toppush}
\setlength{\toppush}{2\headheight}
\addtolength{\toppush}{\headsep}


\def\subjnum{Comp 170}
\def\subjname{Computation Theory}


\def\doheading#1#2#3{\vfill\eject\vspace*{-\toppush}%
  \vbox{\hbox to\textwidth{{\bf} \subjnum: \subjname \hfil Erli Cai}%
    \hbox to\textwidth{{\bf} Tufts University, Fall 2020 \hfil#3\strut}%
    \hrule}}


\newcommand{\htitle}[1]{\vspace*{1.25ex plus 1ex minus 0ex}%
\begin{center}
{\large\bf #1}
\end{center}} 



\begin{document}
\doheading{2}{title}{Homework 07}

\setlength\parindent{0pt}


\section{Out of Context}
a.
$S \rightarrow
\begin{bmatrix}
b\\
b\\
q_s
\end{bmatrix}
\begin{bmatrix}
\$\\
\$\\
\quad
\end{bmatrix}$, \quad
$\begin{bmatrix}
b\\
b\\
q_s
\end{bmatrix}
\rightarrow
\begin{bmatrix}
b\\
b\\
q_s
\end{bmatrix}
\begin{bmatrix}
a\\
a\\
\quad
\end{bmatrix}
$
for all $a, b\in \Sigma$\\

b. Now we try to simulate M ( based on $\delta: Q \times \Gamma \rightarrow Q \times \Gamma\times\{L.R\}$)\\
if we have $\delta(q,A) = (p,B,L)$,then we add in our production rule:\\
$
\begin{bmatrix}
c\\
C\\
\quad
\end{bmatrix}
\begin{bmatrix}
a\\
A\\
q
\end{bmatrix}
\rightarrow
\begin{bmatrix}
c\\
C\\
p
\end{bmatrix}
\begin{bmatrix}
a\\
B\\
\quad
\end{bmatrix}
$
for all $c,C \in \Gamma$

if we have $\delta(q,A) = (p,B,L)$,then we add in our production rule:\\
$
\begin{bmatrix}
a\\
A\\
q
\end{bmatrix}
\begin{bmatrix}
c\\
C\\
\quad
\end{bmatrix}
\rightarrow
\begin{bmatrix}
a\\
B\\
\quad
\end{bmatrix}
\begin{bmatrix}
c\\
C\\
p
\end{bmatrix}
$
for all $c,C \in \Gamma$\\

c.
$
\begin{bmatrix}
n\\
N\\
\quad
\end{bmatrix}......
\begin{bmatrix}
a\\
A\\
\quad
\end{bmatrix}
\begin{bmatrix}
b\\
B\\
q_{accept}
\end{bmatrix}
\begin{bmatrix}
c\\
C\\
\quad
\end{bmatrix}......
\begin{bmatrix}
z\\
Z\\
\quad
\end{bmatrix}
\rightarrow n....abc....z
$
This is the only production rule that doesn't generate non-terminate, so grammar produce a string x if and only if M reaches accepted state

\pagebreak
\section{Tough Decisions}
a.Given DFA D, $ \{<D> |$ L(D) is finite$\}$ is decidable.\\
Suppose D has n states, TM search for all the path of n+1 steps. If any state can be visited repeatedly is one path, then mark that state. If in any path, any accepted state is reached passing through a marked state, then we reject. Else, accept.\\

b. This is recognisable but not decidable. Recognisable because, we can simulate M on all possible strings and accept if any halt.\\ It is not decidable. Because if it is decidable we know when TM never halts and then $\{<M,x>| M is Turing machine, and M accept x\}$ would be decidable (which we have shown in class is not decidable).\\

c. This is not recognisable. Since M could loop on x, we never know if it will read all of x, so not recognisable.





\pagebreak
\section{Closure Properties}
a. This won't work. Suppose $M_A$ loops on x, while $M_B$ accepts x.  since $M_A$ loops on x, we never get to step 2. So the TM never halts while it should accept x.\\
We could build the Turing machine do the following, Alternate simulate $M_A, M_B$ on x, if one of them accept, then accept.\\

b. This will work. Since both machines ($T_A,T_B$) need to halt and accept x for x to be accepted by our new machine. If either machine loops or rejects, the new machine won't accept.









\end{document}


