
\documentclass[12pt]{article}

\usepackage{graphicx}
\graphicspath{ {./images/} }

\usepackage{epsfig}
\usepackage{amsmath,amsthm}
\usepackage{listings}


\newtheorem{lemma}{Lemma}
\newtheorem{theorem}{Theorem}


\usepackage{titlesec}
\titleformat{\section}
{\normalfont\Large\bfseries}{Question~\thesection:}{1em}{}

\newlength{\toppush}
\setlength{\toppush}{2\headheight}
\addtolength{\toppush}{\headsep}


\def\subjnum{Comp 170}
\def\subjname{Computation Theory}


\def\doheading#1#2#3{\vfill\eject\vspace*{-\toppush}%
  \vbox{\hbox to\textwidth{{\bf} \subjnum: \subjname \hfil Erli Cai}%
    \hbox to\textwidth{{\bf} Tufts University, Fall 2020 \hfil#3\strut}%
    \hrule}}


\newcommand{\htitle}[1]{\vspace*{1.25ex plus 1ex minus 0ex}%
\begin{center}
{\large\bf #1}
\end{center}} 

\setlength\parindent{0pt}

\begin{document}
\doheading{2}{title}{Homework 08}

\section{Totally}
a. Rice Theorem does not apply. \\
Suppose we have have two Turing Machines $M_1, M_2$ with $M_1$ rejects all strings and $M_2$ loops on all strings, then $L(M_1) = L(M_2)$.\\
$M_1$ will halt on all input, while $M_2$ does not halt on any input\\
Thus this is not a property of language and Rice's theorem does not apply.\\

b.\\
First we show there exist function f from H to TOT,\\
$f=$ ``on input $\langle M,x\rangle$:
\begin{enumerate}
    \item compute/construct a TM $\hat{M}$ such that $\langle M,x \rangle \in H \iff \langle \hat{M} \rangle \in TOT$:

$\hat{M}=$`` on input $y$:
\vspace*{-10pt}
\begin{itemize}
\item[1.] Ignore y for now
\item[2.] Simulate $M$ on $x$; 
\begin{enumerate}
    \item[1.] If $M$ halts on $x$, $\hat{M}$ \emph{accepts} $y$;
    \item[2.] If $M$ does not halt on $x$, send $\hat{M}$ into a loop state.''
\end{enumerate}
\end{itemize}
\item Return $\langle \hat{M} \rangle$. ''
\end{enumerate}

We claim that M halts on x if and only if $\hat{M}$ halts on all input
\begin{align*}
\langle M,x \rangle \in H & \Rightarrow  M \textrm{ accepts $x$} & \langle M,x \rangle \not\in {\sf A}_{TM} & \Rightarrow  M \textrm{ does not accept $x$}\\ 
& \Rightarrow  M \textrm{ halts on x } & & \Rightarrow  M \textrm{ does not halt on input $x$}\\
& \Rightarrow  \hat{M} \textrm{ accepts all input y } & & \Rightarrow  \hat{M} \textrm{ loops on all input $y$}\\
& \Rightarrow  \langle \hat{M}\rangle \in {\sf TOT}& & \Rightarrow \langle \hat{M} \rangle \not \in {\sf TOT}~.
\end{align*}
%\vspace*{-20pt}

Thus, we have shown that $H \le_m TOT$.


\pagebreak
\section{Not Totally}

$\overline{TOT} =  \{\langle M \rangle ~|~ M \textrm{ is a TM and } M \textrm{ doesn't halt on some inputs}\}.$   \\
First we show there exist function f from H to $\overline{TOT}$,\\
$f=$ ``on input $\langle M,x\rangle$:
\begin{enumerate}
    \item compute/construct a TM $\hat{M}$ such that $\langle M,x \rangle \in H \iff \langle \hat{M} \rangle \in \overline{TOT}$:

$\hat{M}=$`` on input $y$:
\vspace*{-10pt}
\begin{itemize}
\item[1.] Ignore y for now
\item[2.] Simulate $M$ on $x$; 
\begin{enumerate}
    \item[1.] If $M$ halts on $x$, send $\hat{M}$ into a loop state;
    \item[2.] If $M$ does not halt on $x$, $\hat{M}$ \emph{accepts} $y$.''
\end{enumerate}
\end{itemize}
\item Return $\langle \hat{M}\rangle$. ''
\end{enumerate}

We claim that $\langle M, x\rangle \in H$ if and only if $\langle\hat{M}\rangle \in \overline{TOT}$
\begin{align*}
\langle M,x \rangle \in H & \Rightarrow  M \textrm{ accepts $x$} & \langle M,x \rangle \not\in {\sf A}_{TM} & \Rightarrow  M \textrm{ does not accept $x$}\\ 
& \Rightarrow  M \textrm{ halts on x } & & \Rightarrow  {M} \textrm{ does not halt on x }\\
& \Rightarrow  \hat{M} \textrm{ loop on all input y } & & \Rightarrow  \hat{M} \textrm{ accept all input $y$}\\
& \Rightarrow  \langle \hat{M}\rangle \in {\sf \overline{TOT}}& & \Rightarrow \langle \hat{M} \rangle \not \in {\sf \overline{TOT}}~.
\end{align*}
%\vspace*{-20pt}

Thus, we have shown that $H \le_m \overline{TOT}$.




\pagebreak
\section{Differences}

a.  $A\setminus A'$ is decidable.\\
Since $A, A'$ are decidable, there exists $M_1, M_2$ such that $L(M_1) = A, L(M_2) = A'$.\\
Construct $M_3$ in the following way:\\
$M_3$ on input x:\\
Simulate $M_2$ on x, \\
if $M_2$ accepts x, then reject;\\
If $M_2$ rejects x, then simulate $M_1$ on x, if $M_1$ accepts x, then accept,  if $M_1$ rejects x, then reject\\

By construction $L(M_3) = A\setminus A'$
Since $M_1, M_2$ are decidable, we are guaranteed that $M_1, M_2$ will halt. So $M_3$ will halt. So $A\setminus A'$ is decidable.\\

b.
$A\setminus B$  is not even recognisable.\\
We will show it by contradiction:\\
Let $M_1$ be a Turing machine that decides A.\\ 
For the sake of contradiction, suppose we have $M_3$ that recognise $A\setminus B$ .\\
Then construct $M_2$ in the following way\\
$M_2$ on input x:\\
Simulate $M_1$ on x, if reject, then reject.\\
If $M_1$ accept x, then simulate $M_3$ on x, if $M_3$ accept x, accept x.\\
By construction $M_2$ will accept $\overline B$.\\

Since both B and $\overline{B}$ are recognisable, we know that B is decidable (Contradiction)\\
Therefore, our original assumption ($A\setminus B$  is recognisable) is False\\

c.  $B\setminus A$ is recognisable but not decidable.\\
Given A, B there exists $M_1, M_2$ such that $L(M_1) = A, L(M_2) = B'$.\\
Construct $M_3$ in the following way\\
$M_3$ on input x:\\
Simulate $M_2$ on x, \\
if $M_2$ rejects x, then reject;\\
If $M_2$ accepts x, then simulate $M_1$ on x, if $M_1$ accepts x, then reject,  if $M_1$ rejects x, then accept\\

By construction $L(M_3) = B\setminus A$
Since $B$ is s recognisable but not decidable, It might loop when we simulate $M_2$ on x.So $M_3$ may not halt and $ B\setminus A$ is recognisable but not decidable.\\




\end{document}


